\documentclass[titlepage, a4paper,12pt]{jarticle}
\usepackage{listings}
\lstset{
  breaklines=true,
}
\usepackage{comment}
\usepackage{moreverb}
\usepackage[dvipdfmx]{graphicx}
\title{平成26年度 3回生前期学生実験SW \\ 最終レポート}
\author{松田貴大 \\ \small{学籍番号:1029-24-4015}}
\date{提出日:\today 17:00 \\ 提出期限:平成26 年7 月25 日 17:00}
%%
\makeatletter
\renewcommand{\thefigure}{\thesection.\arabic{figure}}
\renewcommand{\thetable}{\thesection.\arabic{table}}
\@addtoreset{figure}{section}
\makeatother
%%
\begin{document}
\maketitle
\lstset{numbers=left,basicstyle=\small}
\section{はじめに}
使用言語はRuby(2.1.0p0)で、パーサの作成にはracc(1.4.11)\footnote{http://i.loveruby.net/ja/projects/racc/}を用いている。
%%
%%
%%
\section{ソースコード}
\label{sec:source}
パスは、\~{}/2014sw/compiler/report3/compiler.yである。プログラムをレポートに載せると膨大になるので、やめておく。
%% \lstinputlisting{compiler.y}
%%
%%
\section{対象文法の変更}
言語仕様を拡張し、対象となる文法を変更したので、以下に記述する。
\subsection{文法}
\begin{verbatim}
  program
      : external_declaration
      | program external_declaration
  external_declaration
      : declaration
      | function_definition
  declaration
      : DATATYPE declarator_list ';'
  declarator_list
      : declarator
      | declarator_list ',' declarator
  declarator
      : IDENTIFIER
  function_definition
      : DATATYPE IDENTIFIER '(' parameter_type_list_opt ')'
        compound_statement
  parameter_type_list
      : parameter_declaration
      | parameter_type_list ',' parameter_declaration
  parameter_type_list_opt
      : parameter_type_list
      | /* none */
  parameter_declaration
      : DATATYPE declarator
  statement
      : ';'
      | 'continue' ';'
      | 'break' ';'
      | expression ';'
      | compound_statement
      | 'if' '(' expression ')' statement
      | 'if' '(' expression ')' statement 'else' statement
      | 'while' '(' expression ')' statement
      | 'for' '(' expression ';' expression ';' expression ')'
        statement
      | 'return' expression ';'
  compound_statement
      : '{' declaration_list_opt statement_list_opt '}'
  declaration_list
      : declaration
      | declaration_list declaration
  declaration_list_opt
      : declaration_list
      | /* none */
  statement_list
      : statement
      | statement_list statement
  statement_list_opt
      : statement_list
      | /* none */
  expression
      : assign_expr
      | expression ',' assign_expr
  assign_expr
      : logical_OR_expr
      | IDENTIFIER '=' assign_expr
      | IDENTIFIER '+=' assign_expr
      | IDENTIFIER '-=' assign_expr
      | IDENTIFIER '*=' assign_expr
      | IDENTIFIER '/=' assign_expr
      | IDENTIFIER '%=' assign_expr
  logical_OR_expr
      : logical_AND_expr
      | logical_OR_expr '||' logical_AND_expr
  logical_AND_expr
      : or_expr
      | logical_AND_expr '&&' or_expr
  or_expr
      : xor_expr
      | or_expr '|' xor_expr
  xor_expr
      : and_expr
      | xor_expr '^' and_expr
  and_expr
      : equality_expr
      | and_expr '&' equality_expr
  equality_expr
      : relational_expr
      | equality_expr '==' relational_expr
      | equality_expr '!= relational_expr
  relational_expr
      : add_expr
      | relational_expr '<' add_expr
      | relational_expr '>' add_expr
      | relational_expr '<=' add_expr
      | relational_expr '>=' add_expr
  add_expr
      : mult_expr
      | add_expr '+' mult_expr
      | add_expr '-' mult_expr
  mult_expr
      : unary_expr
      | mult_expr '*' unary_expr
      | mult_expr '/' unary_expr
      | mult_expr '%' unary_expr
  unary_expr
      : posifix_expr
      | '-' unary_expr
  posifix_expr
      : subprimary_expr
      | IDENTIFIER '(' argument_expression_list_opt ')'
  subprimary_expr
      : primary_expr
      | primary_expr '++'
      | primary_expr '--'
  primary_expr
      : IDENTIFIER
      | CONSTANT
      | '(' expression ')'
  argument_expression_list
      : assign_expr
      | argument_expression_list ',' assign_expr
  argument_expression_list_opt
      : argument_expression_list
      | /* none */
\end{verbatim}
\subsection{拡張点}
具体的に、以下の仕様を加えた。
\subsubsection{ビット演算子}
AND、OR、XORのビット演算をそれぞれ\verb|'&'、'|'、'^'|を用いて導入した。意味解析は他の算術演算と同様に行うことが出来る。コード生成では、それぞれアセンブリ命令のAND、OR、XORを用いることで対応出来る。
\subsubsection{剰余演算子}
剰余演算子\verb|'%'|を導入した。意味解析は他の算術演算と同様に行うことが出来る。コード生成では、idivを用いるとedxに剰余が保存されるので、それを用いることで対応出来る。
\subsubsection{複合代入演算子}
複合代入演算子\verb|'+='、'-='、'*='、'/='、'%='|を導入した。たとえば、\verb|a+=5;|という命令では、aに5が加算される。すなわち、\verb|a=a+5;|と同じ演算となる。意味解析は、代入演算と算術演算のものを組み合わせることで行うことが出来る。コード生成では、代入演算と算術演算のコードを生成することで対応出来る。
\subsubsection{インクリメント、デクリメント}
インクリメント\verb|++|、デクリメント\verb|--|を追加した。たとえば、\verb|a++;|という命令は、\verb|a+=1;|と等しい。意味解析は、複合代入演算子のもので行うことが出来る。コード生成では、複合代入演算子のオペランドが1になっているものとすれば対応出来る。
\subsubsection{for}
ループ文forを導入した。
\verb|for(exp1;exp2;exp3) statement|で、\\ 
\verb|exp1;while(exp2){statement;exp3;}|と同じ意味を表す。意味解析は、代入演算とwhileと算術演算のものを組み合わせることで行うことが出来る。コード生成では、代入演算とwhileと算術演算のコードを生成することで対応出来る。
\subsubsection{continue、break}
ループ脱出のためのcontinue、breakを導入した。continueは、その時点でループブロック内のstatementを終了し、for文の場合はexp3を実行した後にループ条件のチェックに戻る。breakは、その時点でループブロックを脱出する。意味解析については\ref{sec:meaning}節を参照のこと。コード生成については\ref{sec:code}節を参照のこと。
\subsubsection{コメント}
\verb|'//'|以降の1行および\verb|'/*A*/'|の形式のA部分をコメントとみなす処理を構文解析時に追加した。これは、以下のようなアルゴリズムで実現できる。
\begin{enumerate}
\item flagを0とする。
\item flagが0のとき、
  \begin{enumerate}
  \item '//'を読んだらflagを1とする。
  \item '/*'を読んだらflagを2とする。
  \item それ以外のときは通常の構文解析を行う。
  \end{enumerate}
\item flagが1のとき、
  \begin{enumerate}
  \item '\verb|\n|'を読んだらflagを0とする。
  \item それ以外のときは無視する。
  \end{enumerate}
\item flagが2のとき、
  \begin{enumerate}
  \item '\verb|*/|'を読んだらflagを0とする。
  \item それ以外のときは無視する。
  \end{enumerate}
\item 2.以降を繰り返す。
\end{enumerate}
%%
%%
\section{意味解析(課題7)}
\label{sec:meaning}
意味解析では、構文解析によって生成された構文木に現れるシンボルが具体的にどのアドレスに存在するものか等を解析し、矛盾が無いかを解析すると共にシンボルの意味を確定させていく。
\subsection{設計方針}
意味解析時に、シンボルに付加できる意味が存在しなかったり曖昧であったりする場合や、continueとbreakがループブロック内に存在しない場合はエラーを出して構文木を破棄する\footnote{エラー行の表示を行いたかったが、生成規則のアクション部分で意味解析を行っていくためエラー行の確定が難しかったので、今回はエラー内容の表示のみを行う。}。また、シンボル宣言時に、異なる深さレベルの名前空間に既に同じ名前のシンボルが存在する場合は警告を表示する。

名前空間を保持するために、スタックを導入する。スタック内のデータは、シンボル名・深さレベル・種類・備考の情報を持つ。あるシンボルが宣言されると、スタックにシンボル情報が積まれる。ある深さレベルの意味解析が終了すると、終了した深さレベルを持つデータをスタックから取り出し、破棄する。スタック内のデータは、必ず深さレベルの小さい順に積まれているので、この方法で意味解析が可能である。
\subsection{解析方法}
以下では、構文木の各ノードからどのように意味解析を行っていくかを説明する。
%%
\subsubsection{宣言文}
関数や変数の宣言文を表すノードは先頭要素が\verb|'int'|から始まる配列である。具体的には、以下のようになっている\footnote{以降、配列をRuby形式で記述する。}。
\begin{itemize}
\item 変数 … \verb|["int", IDENTIFIER]|
\item 関数 … \verb|["int", IDENTIFIER, parameters, statements]|
\end{itemize}
宣言文は、配列の2番目の要素がシンボル名となっている。このシンボル名と同じ名前を持つデータがスタック内に存在するか確認する。データが存在する場合、現在の深さレベルとデータの深さレベルが等しい場合はエラーを出す。異なる場合は警告を出す。データが存在しない場合、シンボルが変数ならスタック状態からアドレスを確定させて構文木内のシンボルに情報を付加して備考とし、シンボルが関数なら引数の数を備考とした後、スタックにシンボルデータを積む。

なお、実験資料ではパラメータ宣言と変数宣言と関数宣言を分けて考えているが、特にその必要性は無いと判断したのでパラメータ宣言や関数宣言の意味解析も変数宣言に含んでいる\footnote{関数宣言は必ず深さレベル0で行われるため、二重宣言の際は必ずエラー検出が可能である。}。
%%
\subsubsection{代入演算}
代入演算を表すノードは先頭要素が\verb|'='|から始まる配列である。具体的には、以下のようになっている。
\begin{verbatim}
["=", IDENTIFIER, expression]
\end{verbatim}
代入演算は、配列の2番目の要素が代入する変数名となっている。この変数名を\ref{sec:ref}節の方法で解析する。
%%
\subsubsection{continue、break}
continue、breakのノードは、それぞれ\verb|['CONTINUE']|、\verb|['BREAK']|である。以下のアルゴリズムで、continue・breakが適切な位置で使用されているか解析出来る。
\begin{enumerate}
\item ループの深さを保持する変数loop\_depthを用意して、0を代入する。
\item FOR、WHILE文のループブロックに入るとき、loop\_depthに1を足す。
\item ループブロックを抜けるとき、loop\_depthから1を引く。
\item continueまたはbreakが現れたとき、loop\_depthが0ならエラー。
\item 2.以降を繰り返す。
\end{enumerate}
%%
\subsubsection{変数参照}
\label{sec:ref}
変数名と同じ名前を持つ変数データがスタック内に存在するか確認する。データが存在しない場合、エラーを出す。データが存在する場合、スタック上で最も近い位置に存在する同じ名前のデータを参照する変数として、構文木内の変数にアドレス情報を付加する。
%%
\subsubsection{関数呼び出し}
関数呼び出しを表すノードは先頭要素が\verb|'FCALL'|から始まる配列である。具体的には、以下のようになっている。
\begin{verbatim}
["FCALL", IDENTIFIER, parameter_list]
\end{verbatim}
関数名と同じ名前を持つ関数データがスタック内に存在するか確認する。データが存在しない場合、未定義関数の呼び出しとして警告を出す。データが存在する場合、引数の数を照合して一致しない場合はエラーを出す。
%%
\subsection{実行例と結果}
\ref{sec:source}節で述べたソースコードから生成されたパーサと構文木生成プログラムを使い、いくつかのTinyCプログラムの構文を解析した後に意味解析を行った結果を示す。
%%
\subsubsection{例1}
\lstinputlisting{test.tc}
\paragraph{結果}
\begin{verbatim}
warning: Variable 'x' is already defined at level 0.
warning: Variable 'x' is already defined at level 1.
warning: Variable 'x' is already defined at level 2.
warning: Variable 'y' is already defined at level 1.
warning: Variable 'x' is already defined at level 3.
error: Parameter-length of function 'f' is 2.
\end{verbatim}
%%
\subsubsection{例2}
\lstinputlisting{test2.tc}
\paragraph{結果}
\begin{verbatim}
error: break is NOT available here.
error: continue is NOT available here.
\end{verbatim}
%%
%%
\section{コード生成(課題8)}
\label{sec:code}
意味解析でエラーが発生しなかった場合、アドレス情報の付加された構文木を用いてコード生成を行っていく。
\subsection{設計方針}
命令・文の種類はノードの先頭要素で判断出来るので、それを用いて分岐処理を行い、コード生成を行う。
\subsection{生成方法}
以下では、コード生成方法のうち資料と異なるものについて述べる。
\subsubsection{IF文}
IF文を表すノードは以下のようになっている。
\begin{verbatim}
["IF", expression, statement1, statement2]
\end{verbatim}
expressionは真偽値を返す。真の場合はstatement1、偽の場合はstatement2を実行する。すなわち、
\begin{verbatim}
if(expression)  statement1
else            statement2
\end{verbatim}
に対応する。elseが無い場合、statement2は空の配列になる。

コード生成については資料と同じだが、ラベル名は以下の規則で付ける。
\begin{itemize}
\item else対応部分 … {関数名}\_else{関数内でIF文が出現した回数}
\item IF文終了対応部分 … {関数名}\_if{関数内でIF文が出現した回数}
\end{itemize}
%%
\subsubsection{WHILE文}
コード生成については資料と同じだが、ラベル名は以下の規則で付ける。
\begin{itemize}
\item WHILE文開始対応部分 … {関数名}\_whilestart{関数内でWHILE文・FOR文が出現した回数}
\item WHILE文終了対応部分 … {関数名}\_whileend{関数内でWHILE文・FOR文が出現した回数}
\end{itemize}
%%
\subsubsection{FOR文}
コード生成については資料と同じだが、ラベル名は以下の規則で付ける。
\begin{itemize}
\item FOR文開始対応部分 … {関数名}\_whilestart{関数内でWHILE文・FOR文が出現した回数}
\item FOR文ループ毎処理対応部分\footnote{continue時に使用する。} … {関数名}\_whilemid{関数内でWHILE文・FOR文が出現した回数}
\item FOR文終了対応部分 … {関数名}\_whileend{関数内でWHILE文・FOR文が出現した回数}
\end{itemize}
%%
\subsubsection{continue、break}
continue時には、対応ループのmidラベルにジャンプする。break時には、対応ループのendラベルにジャンプする。
%%
%%
\subsection{最適化}
コード生成後、以下の最適化を行う。生成したコードは、まず文字列として変数に保存する。その後、文字列を走査して以下の条件に当てはまる部分が存在すれば、最適化操作を行う。最適化後に新たな最適化可能部分が生成される可能性もあるため、最適化が行われなくなるまで走査を繰り返す。
\subsubsection{無意味な無条件ジャンプ}
たとえば、以下のようなアセンブリコードではjmp命令は不必要であるので、削除する。
\begin{verbatim}
    jmp L1
L1:
\end{verbatim}
このような無意味な無条件ジャンプは、return文生成時によく現れる。
%%
\subsubsection{無意味なラベル}
コード生成時には、IF文にelseが存在しない場合でもelseラベルを生成する。すなわち、IF文にelseが存在しない場合は以下のようなコードが生成される。
\begin{verbatim}
    je f_else0
;   ### 条件式が真 ###
    jmp f_if0
f_else0:
f_if0:
\end{verbatim}
これは、以下のようなコードに書き換えられる。
\begin{verbatim}
    je f_if0
;   ### 条件式が真 ###
    jmp f_if0
f_if0:
\end{verbatim}
このように、不必要なラベルを消去することが出来る。
%%
\subsubsection{実行されない命令}
無条件ジャンプからラベルまでの処理は、実行されることはない。
\begin{verbatim}
    jmp L1
    add eax, ebx
    sub ebx, eax
L:
\end{verbatim}
このようなコードでは、加算命令addと減算命令subは実行されることはない。よって、この部分を削除して
\begin{verbatim}
    jmp L1
L:
\end{verbatim}
と書き換えられる。
%%
\subsubsection{関数定義時のesp操作}
まだ作ってないよ。
%%
\subsection{実行例と結果}
\ref{sec:source}節で述べたソースコードから生成されたパーサと構文木生成プログラムを使い、いくつかのTinyCプログラムの構文を解析した後に意味解析を行い、最適化の後コード生成を行った結果を示す。
なお、実行するプログラムは共通して以下のものである。
\paragraph{mainプログラム}
\lstinputlisting{main.c}
\subsection{例1}
\lstinputlisting{test3.tc}
\paragraph{出力アセンブリコード}
\begin{verbatimtab}[4]
	GLOBAL	f
f:
	push	ebp
	mov	ebp, esp
	sub	esp, 0
	mov	dword eax, 0
	push	eax
	mov	eax, [ebp+8]
	pop	ebx
	cmp	eax, ebx
	setle	al
	movzx	eax, al
	cmp	eax, 0
	je	f_else0
	mov	eax, 1
	jmp	f_ret
f_else0:
	mov	dword eax, 1
	push	eax
	mov	eax, [ebp+8]
	pop	ebx
	sub	eax, ebx
	push	eax
	call	f
	pop	ebx
	push	eax
	mov	eax, [ebp+8]
	pop	ebx
	imul	eax, ebx
	jmp	f_ret
f_if0:
f_ret:
	mov	esp, ebp
	pop	ebp
	ret
\end{verbatimtab}
\paragraph{結果}
\begin{verbatim}
1
1
2
6
24
120
720
5040
40320
362880
\end{verbatim}
%%
\subsection{例2}
\lstinputlisting{test4.tc}
\paragraph{出力アセンブリコード}
\begin{verbatimtab}[4]
	GLOBAL	f
f:
	push	ebp
	mov	ebp, esp
	sub	esp, 8
	mov	dword [ebp-8], 0
	mov	dword [ebp-4], 1
f_whilestart0:
	mov	eax, [ebp+8]
	push	eax
	mov	eax, [ebp-4]
	pop	ebx
	cmp	eax, ebx
	setle	al
	movzx	eax, al
	cmp	eax, 0
	je	f_whileend0
	mov	eax, [ebp-4]
	push	eax
	mov	eax, [ebp-8]
	pop	ebx
	add	eax, ebx
	mov	[ebp-8], eax
f_whilemid0:
	mov	dword eax, 1
	push	eax
	mov	eax, [ebp-4]
	pop	ebx
	add	eax, ebx
	mov	[ebp-4], eax
	jmp	f_whilestart0
f_whileend0:
	mov	eax, [ebp-8]
f_ret:
	mov	esp, ebp
	pop	ebp
	ret
\end{verbatimtab}
\paragraph{結果}
\begin{verbatim}
0
1
3
6
10
15
21
28
36
45
\end{verbatim}

%%
%%
%%
\section{感想}
コード生成部分に非常に時間がかかった。バグが発生した場合、アセンブリコードのどの部分がおかしいか特定するのが非常に難しかった。拡張は文法規則から変更するのでいろいろと考えるべきことが多かったが、楽しかった。最適化は、様々な手法が考えられるが、結局キリが無いような気がした。今回の実験を通して、コンパイラが行っていることを理解できた。
\end{document}
