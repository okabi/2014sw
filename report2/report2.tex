\documentclass[titlepage, a4paper,12pt]{jarticle}
\usepackage{listings}
\lstset{
  breaklines=true,
}
\usepackage{comment}
\usepackage[dvipdfmx]{graphicx}
\title{平成26年度 3回生前期学生実験SW \\ 中間レポート2}
\author{松田貴大 \\ \small{学籍番号:1029-24-4015}}
\date{提出日:\today 17:00 \\ 提出期限:平成26 年6 月27 日 16:00}
%%
\makeatletter
\renewcommand{\thefigure}{\thesection.\arabic{figure}}
\renewcommand{\thetable}{\thesection.\arabic{table}}
\@addtoreset{figure}{section}
\makeatother
%%
\begin{document}
\maketitle
\lstset{numbers=left,basicstyle=\small}
\section{はじめに}
使用言語はRuby(2.1.0p0)で、パーサの作成にはracc(1.4.11)\footnote{http://i.loveruby.net/ja/projects/racc/}を用いている。
%%
%%
\section{課題6}
\subsection{ソースコード}
\label{sec:source}
パスは、\~{}/2014sw/compiler/report2/compiler.yである。
\lstinputlisting{compiler.y}
%%
\subsection{設計方針}
\label{sec:arch}
実験資料とは少し異なる仕様の構文木を生成する。たとえば、実験資料の例(課題7のもの)を見ると、以下のような構文木が生成されている\footnote{以下では、構文木をRuby形式の配列で記述する。}。
\paragraph{TinyCプログラム}
\begin{verbatim}
int x;
int f(int x, int y)
{
  int x;
  {
    int x, y;
    x+y;
    {
      int x, z;
      x+y+z;
    }
  }
  {
    int w;
    x+y+w;
  }
  x+y;
}
int g(int y)
{
  int z;
  f(x, y);
  g(z);
}
\end{verbatim}
\paragraph{構文木}
\begin{verbatim}
[int x]
[[int f] [[int x][int y]]
[
  [int x]
  [
    [int x y]
    [+ x y]
    [
      [int x z]
      [+ [+ x y] z]
    ]
  ]
  [
    [int w]
    [+ [+ x y] w]
  ]
  [+ x y]
]]
[[int g] [[int y]]
[
  [int z]
  [FCALL f x y]
  [FCALL g z]
]]
\end{verbatim}
今回採用した仕様では、同じTinyCプログラムは以下のような構文木を生み出す\footnote{実際には改行やインデントは行われない。}。
\paragraph{構文木}
\begin{verbatim}
[
  ["int" "x"]
  [["int" "f"][["int" "x"] ["int" "y"]]
  [
    ["int" "x"]
    ["int" "x"]
    ["int" "y"]
    ["+" "x" "y"]
    ["int" "x"]
    ["int" "z"]
    ["+" ["+" "x" "y"] "z"]
    ["int" "w"]
    ["+" ["+" "x" "y"] "w"]
    ["+" "x" "y"]
  ]]
  [["int" "g"] [["int" "y"]]
  [
    ["int" "z"]
    ["FCALL" "f" ["x" "y"]]
    ["FCALL" "g" ["z"]]
  ]]
]
\end{verbatim}
以下では、資料の仕様から変更した点について述べる。
\paragraph{全体で1つの配列}
大域で複数の変数や関数が宣言されている場合、全体を1つの配列にまとめるようにした。これは、後にこのプログラム全体を処理する際に、処理を楽にするためである。
\paragraph{複数変数宣言}
\subparagraph{TinyCプログラム}
\begin{verbatim}
int x, y;
\end{verbatim}
以上のように、データ型宣言の後に複数変数を宣言した場合、資料では以下のように構文木を生成している。
\subparagraph{構文木}
\begin{verbatim}
[int x y]
\end{verbatim}
この設計だと、変数宣言時に配列の長さが可変(2以上)になってしまい、処理がやや面倒になる。そのため、複数変数宣言されている場合、以下のように別々の配列を生成するようにした。
\subparagraph{構文木}
\begin{verbatim}
["int" "x"]
["int" "y"]
\end{verbatim}
\paragraph{不必要な括弧}
資料では、スコープを制限するための中括弧('\{'や'\}')によって、構文木に括弧が生成される。しかし、raccでは構文木の生成と同時にスコープの深さを調べることが出来る(すなわち、構文木生成と同時に意味解析が完全に可能である)ので、これらの括弧は不必要となる。よって、不必要な括弧は省略するようにした\footnote{もちろん、ifやwhileなどでは範囲を括弧で囲む。}。
\paragraph{関数呼び出し時の引数}
\subparagraph{TinyCプログラム}
\begin{verbatim}
f(x, y);
\end{verbatim}
以上のように、関数呼び出しの際に引数が複数ある場合、資料では以下のように構文木を生成している。
\subparagraph{構文木}
\begin{verbatim}
[FCALL f x y]
\end{verbatim}
この設計だと、関数呼び出し時に配列の長さが可変(2以上)になってしまい、処理がやや面倒になる。そのため、どのような場合でも引数全体を配列で包み、関数呼び出し部分の配列の長さは3になるようにした。
\subparagraph{構文木}
\begin{verbatim}
["FCALL" "f" ["x" "y"]]
\end{verbatim}
同様に、以下のようなコードは、以下のような構文木になる。
\subparagraph{TinyCプログラム}
\begin{verbatim}
g(int x);
h();
\end{verbatim}
\subparagraph{構文木}
\begin{verbatim}
["FCALL" "g" ["x"]]
["FCALL" "h" []]
\end{verbatim}
%%
\subsection{文法規則のアクション部分について}
%%
\subsection{演算子の優先順位が正しく反映される理由}
%%
\subsection{工夫点}
節\ref{sec:arch}で述べたように、実験資料とは異なる構造の構文木を生成している。
%%
\subsection{実行例と結果}
\ref{sec:source}節で述べたソースコードから生成されたパーサと構文木生成プログラムを使い、いくつかのTinyCプログラムの構文を解析した結果を示す。
\subsubsection{例1}
\lstinputlisting{test.tc}
\paragraph{結果}
\begin{verbatim}
success!!! 
 result => int
\end{verbatim}
%%
%%
\section{感想}

\end{document}
